\documentclass[useAMS,usenatbib]{mn2e}

\def\aap{AA}
\def\aapr{AA Rev}
\def\apjl{ApJL}
\def\apss{APSS}
\def\mnras{MNRAS}
\def\apj{ApJ}
\def\apjs{ApJS}
\def\aj{AJ}
\def\pasp{PASP}
\def\pasj{PASJ}
\def\nat{Nat}
\def\memsai{MmSAI}

%\def\RDM{{R_{\rm dm}}}

\usepackage{graphicx}
\usepackage{float}
\usepackage{amssymb}
\usepackage{amsfonts}
\usepackage{amsmath} 
\usepackage{color}
%\usepackage{hyperref}

\def\ucsb{Department of Physics, University of California, Santa Barbara, CA 93106, USA}
\def\ucla{Physics and Astronomy Building, 430 Portola Plaza, Box 951547, Los Angeles, CA 90095-1547, USA}
\def\ioa{Institute of Astronomy, UK}
\def\kipac{Kavli Institute for Particle Astrophysics and Cosmology, Stanford University, 452 Lomita Mall, Stanford, CA 94305, USA}

\def\aaemail{\tt aagnello@physics.ucsb.edu, tt@astro.ucla.edu}
\def\xlmemail{\tt xlmeng919@gmail.com}


\title[Imaging time delay lenses]{Precision cosmology with time delay lenses: high resolution imaging requirements} 
\author[Meng et al.]{
  Xiao-Lei Meng$^{1}$,\thanks{\xlmemail},
  Tommaso Treu$^{1,2}$,
  Adriano Agnello$^{1,2}$,
\newauthor
  Matthew W.~Auger$^{3}$,
  Kai Liao$^{1,2}$,
  Philip J. Marshall$^{4}$,
  et al.
  \medskip\\
  $^1$\ucsb\\
  $^2$\ucla\\
  $^3$\ioa\\
  $^4$\kipac\\
}

\begin{document}

\voffset-.6in

\date{Accepted . Received }

\pagerange{\pageref{firstpage}--\pageref{lastpage}} 

\maketitle

\label{firstpage}

\begin{abstract}
Gravitational time delays are a powerful probe of cosmology, provided
that the gravitational potential of the main deflector can be modeled
with sufficient precision. Recent work has shown that this can be
achieved by detailed modeling of the host galaxies of lensed
quasars. The distortion of the images as measured over large number of
pixels provides tight constraints on the difference between the
gravitational potential between the two quasars, and thus on cosmology
in combination with the measured time delay. We carry out a systematic
exploration of the high resolution imaging required to eploit the
thousands of lensed quasars that will be discovered by current and
upcoming surveys with the next decade. Specifically we simulate
realistic lens systems as imaged by the Hubble Space Telescope, ground
based adaptive optics images taken with Keck or the Thirty Meter
Telescope. We compare the performance of these pointed observations
with that of images taken by the Euclid-VIS and WFIRST surveys. Using
as our metric the precision with which the slope of the mass density
profile for the main deflector can be measured we find that...
\end{abstract}
\begin{keywords}
...
\end{keywords}

\section{Introduction}

\section{Simulations}

{\bf XIAO-LEI: make a montage with the same lensing system shown in various configurations of exposure time and instrument: one row per instrument, one column per exposure time. Only one exposure time for Euclid and WFIRST}

\subsection{Hubble Space Telescope}
\subsubsection{ACS}
\subsubsection{WFC3}
\subsection{Keck 10m Telescope}
\subsubsection{LGSAO-NIRC2}
\subsubsection{NGAO}
\subsection{Thirty Meter Telescope}
\subsubsection{IRIS}
\subsection{Euclid}
\subsection{WFIRST}

\section{Results}

\section{Summary}

\section*{Acknowledgments}



\end{document}
